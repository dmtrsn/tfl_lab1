\documentclass[a4paper, 14pt]{extarticle}

% Поля
%--------------------------------------
\usepackage{geometry}
\geometry{a4paper,tmargin=2cm,bmargin=2cm,lmargin=3cm,rmargin=1cm}
%--------------------------------------


%Russian-specific packages
%--------------------------------------
\usepackage[T2A]{fontenc}
\usepackage[utf8]{inputenc} 
\usepackage[english, main=russian]{babel}
%--------------------------------------

\usepackage{textcomp}

% Красная строка
%--------------------------------------
\usepackage{indentfirst}               
%--------------------------------------             

\newcommand{\T}{\mathcal{T}}

%Graphics
%--------------------------------------
\usepackage{graphicx}
\graphicspath{ {./images/} }
\usepackage{wrapfig}
%--------------------------------------

% Полуторный интервал
%--------------------------------------
\linespread{1.3}                    
%--------------------------------------

%Выравнивание и переносы
%--------------------------------------
% Избавляемся от переполнений
\sloppy
% Запрещаем разрыв страницы после первой строки абзаца
\clubpenalty=10000
% Запрещаем разрыв страницы после последней строки абзаца
\widowpenalty=10000
%--------------------------------------

%Списки
\usepackage{enumitem}

%Подписи
\usepackage{caption} 

%Гиперссылки
\usepackage{hyperref}

\hypersetup {
	unicode=true
}

%Рисунки
%--------------------------------------
\DeclareCaptionLabelSeparator*{emdash}{~--- }
\captionsetup[figure]{labelsep=emdash,font=onehalfspacing,position=bottom}
%--------------------------------------

\usepackage{tempora}


%Листинги
%--------------------------------------
\usepackage{listings}
\lstset{
  basicstyle=\ttfamily\footnotesize, 
  %basicstyle=\footnotesize\AnkaCoder,        % the size of the fonts that are used for the code
  breakatwhitespace=false,         % sets if automatic breaks shoulbd only happen at whitespace
  breaklines=true,                 % sets automatic line breaking
  captionpos=t,                    % sets the caption-position to bottom
  inputencoding=utf8,
  frame=single,                    % adds a frame around the code
  keepspaces=true,                 % keeps spaces in text, useful for keeping indentation of code (possibly needs columns=flexible)
  keywordstyle=\bf,       % keyword style
  numbers=left,                    % where to put the line-numbers; possible values are (none, left, right)
  numbersep=5pt,                   % how far the line-numbers are from the code
  xleftmargin=25pt,
  xrightmargin=25pt,
  showspaces=false,                % show spaces everywhere adding particular underscores; it overrides 'showstringspaces'
  showstringspaces=false,          % underline spaces within strings only
  showtabs=false,                  % show tabs within strings adding particular underscores
  stepnumber=1,                    % the step between two line-numbers. If it's 1, each line will be numbered
  tabsize=2,                       % sets default tabsize to 8 spaces
  title=\lstname                   % show the filename of files included with \lstinputlisting; also try caption instead of title
}

%--------------------------------------

%%% Математические пакеты %%%
%--------------------------------------
\usepackage{amsthm,amsfonts,amsmath,amssymb,amscd}  % Математические дополнения от AMS
\usepackage{mathtools}                              % Добавляет окружение multlined
\usepackage[perpage]{footmisc}
%--------------------------------------

%--------------------------------------
%			НАЧАЛО ДОКУМЕНТА
%--------------------------------------

\begin{document}

%--------------------------------------
%			ТИТУЛЬНЫЙ ЛИСТ
%--------------------------------------
\begin{titlepage}
\thispagestyle{empty}
\newpage


%Шапка титульного листа
%--------------------------------------
\vspace*{-60pt}
\hspace{-65pt}
\begin{minipage}{0.3\textwidth}
\hspace*{-20pt}\centering
\includegraphics[width=\textwidth]{emblem}
\end{minipage}
\begin{minipage}{0.67\textwidth}\small \textbf{
\vspace*{-0.7ex}
\hspace*{-6pt}\centerline{Министерство науки и высшего образования Российской Федерации}
\vspace*{-0.7ex}
\centerline{Федеральное государственное бюджетное образовательное учреждение }
\vspace*{-0.7ex}
\centerline{высшего образования}
\vspace*{-0.7ex}
\centerline{<<Московский государственный технический университет}
\vspace*{-0.7ex}
\centerline{имени Н.Э. Баумана}
\vspace*{-0.7ex}
\centerline{(национальный исследовательский университет)>>}
\vspace*{-0.7ex}
\centerline{(МГТУ им. Н.Э. Баумана)}}
\end{minipage}
%--------------------------------------

%Полосы
%--------------------------------------
\vspace{-25pt}
\hspace{-35pt}\rule{\textwidth}{2.3pt}

\vspace*{-20.3pt}
\hspace{-35pt}\rule{\textwidth}{0.4pt}
%--------------------------------------

\vspace{1.5ex}
\hspace{-35pt} \noindent \small ФАКУЛЬТЕТ\hspace{80pt} <<Информатика и системы управления>>

\vspace*{-16pt}
\hspace{47pt}\rule{0.83\textwidth}{0.4pt}

\vspace{0.5ex}
\hspace{-35pt} \noindent \small КАФЕДРА\hspace{50pt} <<Теоретическая информатика и компьютерные технологии>>

\vspace*{-16pt}
\hspace{30pt}\rule{0.866\textwidth}{0.4pt}
  
\vspace{11em}

\begin{center}
\Large {\bf Лабораторная работа № 1} \\ 
\large {\bf по курсу <<Теория формальных языков>>} \\
% \large «Моделирование данных с использованием модели семантических объектов» 
\end{center}\normalsize

\vspace{8em}


\begin{flushright}
  {Студент группы ИУ9-51Б Санталов Д.А. \hspace*{15pt}\\ 
  \vspace{2ex}
  Преподаватель Непейвода А. Н.\hspace*{15pt}}
\end{flushright}

\bigskip

\vfill
 

\begin{center}
\textsl{Москва 2025}
\end{center}
\end{titlepage}
%--------------------------------------
%		КОНЕЦ ТИТУЛЬНОГО ЛИСТА
%--------------------------------------

\renewcommand{\ttdefault}{pcr}

\setlength{\tabcolsep}{3pt}
\newpage
\setcounter{page}{2}

% Оглавление
\tableofcontents
\newpage

\section{Постановка задачи}\label{Sect::task}
По имеющейся SRS определить:

\begin{enumerate}[topsep=0pt,parsep=-5pt]
    \item завершимость;
    \item конечность классов эквивалентности по НФ (для построения эквивалентностей считаем, что правила могут применяться в обе стороны). Если их конечное число, то построить минимальную систему переписывания, им соответствующую;
    \item локальную конфлюэнтность и пополняемость по Кнуту-Бендиксу.
\end{enumerate}

По SRS $\T$ тем самым (исключая случай, когда она сразу локально конфлюэнтна
или конечна и минимальна) строится другая SRS $\T'$, которая должна сохранять
те же классы эквивалентности. Если исходная SRS завершима, то правила в $\T'$
должны удовлетворять условию убывания левой части относительно правой по выбранному фундированному порядку $\preceq$.

Исходная SRS:

$babb \to bbbab$

$baabb \to babbaab$ 

$baaabb \to baabbaaab$ 

$bbbb \to abab$ 

$aaaa \to a$

\textbf{Фазз-тестирование эквивалентности:} строится случайное слово $\omega$ и случайная цепочка переписываний его в $\omega'$ по $T$. Проверить, можно ли получить $\omega'$ из
$\omega$ (или наоборот) в рамках правил $\T'$.

\textbf{Метаморфное тестирование:} выбрать инварианты, которые должны сохраняться (либо монотонно изменяться) при переписывании в рамках $\T$. Породить случайную цепочку переписываний над случайным словом в $\T'$ и проверить выполнимость инвариантов. 

\section{Основная часть}\label{Sect::realisation}

\subsection{Завершимость}
Исходная SRS незавершима, поскольку есть цикл: $w_1 babb \, bbw_2 \to w_1bb \, babb \, bw_2 \to w_1 bbbb \, babb \, w_2 \to w_1 \, bbbb \, bbbab w_2 \to w_1 a \, babbbb \, ab w_2 \to w_1' babbbb w_2'$.

\subsection{Конечность классов эквивалентности}

Рассмотрим слова: $b(aab)^+$, то есть слова вида: $baab, baabaab, baabaabaab, \ldots$. Поскольку таких слов бесконечно много, и к словам такого вида нельзя применить ни одно из правил, кроме $aaaa \leftrightarrow a$. Но, применяя это правило в обе стороны, не возникнет ситуации, когда мы получим слово, к которому можно будет применить одно из исходных правил (кроме, опять же, $aaaa \leftrightarrow a$). Отсюда: классов эквивалентности бесконечно. 

\subsection{Локальная конфлюэнтность и пополняемость по Кнуту-Бендиксу}

Система не локально конфлюэнтна, так как строка $bbbbb$ может быть редуцирована к двум разным нормальным формам: $bbbbb \to ababb \to abbbab$ и $bbbbb \to babab$.

Пополним исходную систему по алгоритму Кнута-Бендикса и получим эквивалентную систему.

Исходные правила:

$bbbab \to babb$

$babbaab \to baabb$

$baabbaaab \to baaabb$

$bbbb \to abab$

$aaaa \to a$


 \textbf{Шаг 1}.

   • Рассмотрим неразрешимую критическую пару (babbbaab, bbbaabb)

        Добавляем новое правило: $babbbaab \to bbbaabb$

   • Рассмотрим неразрешимую критическую пару (bbbabb, ababbab)

        Добавляем новое правило: $ababbab \to babbb$

   • Рассмотрим неразрешимую критическую пару (bbabb, ababab)

        Добавляем новое правило: $ababab \to bbabb$

   • Рассмотрим неразрешимую критическую пару (babbbbb, bbbaabab)

        Добавляем новое правило: $bbbaabab \to baababb$

   • Рассмотрим неразрешимую критическую пару (baabbbaaab, babbaaabb)

        Добавляем новое правило: $baabbbaaab \to babbaaabb$

   • Рассмотрим неразрешимую критическую пару (baabbbbb, babbaaabab)

        Добавляем новое правило: $babbaaabab \to baaababb$

   • Рассмотрим неразрешимую критическую пару (baaabbbbb, baabbaaaabab)

        Добавляем новое правило: $baabbabab \to bababb$

   • Рассмотрим неразрешимую критическую пару (bbbabab, ababbbb)

        Добавляем новое правило: $abaabab \to babbab$

   • Рассмотрим неразрешимую критическую пару (bbabab, ababbb)

        Добавляем новое правило: $bbabab \to ababbb$

    Обновляем правило: $baabbabab \to bababb$ → $baaababbb \to bababb$

   • Рассмотрим неразрешимую критическую пару (babab, ababb)

        Добавляем новое правило: $babab \to ababb$

    Обновляем правило: $ababab \to bbabb$ → $aababb \to bbabb$

    Удаляем правило: $bbabab \to ababbb$

    Обновляем правило: $baaababbb \to bababb$ → $baaababbb \to ababbb$


 \textbf{Шаг 2}.

   • Рассмотрим неразрешимую критическую пару (ababb, aabbabb)

        Добавляем новое правило: $aabbabb \to ababb$

    Обновляем правило: $bbbaabab \to baababb$ → $bbbaabab \to babbb$

    Обновляем правило: $baaababbb \to ababbb$ → $babbabbb \to ababbb$

    Обновляем правило: $babbaaabab \to baaababb$ → $babbaaabab \to babbabb$

   • Рассмотрим неразрешимую критическую пару (abaabab, aaababbab)

        Добавляем новое правило: $abbabbab \to babbab$

   • Рассмотрим неразрешимую критическую пару (ababbab, aaababbb)

        Добавляем новое правило: $abbabbb \to babbb$

    Обновляем правило: $babbabbb \to ababbb$ → $bbabbb \to ababbb$

   • Рассмотрим неразрешимую критическую пару (aabaabab, bbabbbb)

        Добавляем новое правило: $bbaabab \to babbb$

    Обновляем правило: $abbabbb \to babbb$ → $ababbb \to babbb$

    Обновляем правило: $bbbaabab \to babbb$ → $ababbb \to babbb$

   • Рассмотрим неразрешимую критическую пару (abaabaabab, babbabbbb)

        Добавляем новое правило: $baabbab \to babbb$

    Обновляем правило: $bbabbb \to ababbb$ → $bbabbb \to babbb$

   • Рассмотрим неразрешимую критическую пару (ababbaabab, babbbbbb)

        Добавляем новое правило: $baabab \to babbb$

    Обновляем правило: $abaabab \to babbab$ → $babbb \to babbab$

    Удаляем правило: $bbaabab \to babbb$

   • Рассмотрим неразрешимую критическую пару (babbbaaabab, bbbaabbbbb)

        Добавляем новое правило: $babbbaaabab \to babbb$

    Удаляем правило: $ababbab \to babbb$

    Удаляем правило: $abbabbab \to babbab$

    Обновляем правило: $babbaaabab \to babbabb$ → $babbaaabab \to babbb$

   • Рассмотрим неразрешимую критическую пару (ababbbaab, babaabb)

        Добавляем новое правило: $bbbaabb \to babaabb$

    Обновляем правило: $babbbaab \to bbbaabb$ → $babbbaab \to babaabb$

   • Рассмотрим неразрешимую критическую пару (ababbbbaab, babbbaabb)

        Добавляем новое правило: $babaabb \to abaabbb$

    Обновляем правило: $bbbaabb \to babaabb$ → $bbbaabb \to abaabbb$

    Обновляем правило: $babbbaab \to babaabb$ → $babbbaab \to abaabbb$

   • Рассмотрим неразрешимую критическую пару (bbabbaab, aabaabb)

        Добавляем новое правило: $aabaabb \to bbaabb$

   • Рассмотрим неразрешимую критическую пару (bbabbbaab, aabbbaabb)

        Добавляем новое правило: $abbaabbb \to abaabbb$

   • Рассмотрим неразрешимую критическую пару (babbabbaab, abaabaabb)

        Добавляем новое правило: $bbaabbb \to abaabbb$

    Удаляем правило: $abbaabbb \to abaabbb$

   • Рассмотрим неразрешимую критическую пару (babbabbbaab, abaabbbaabb)

        Добавляем новое правило: $abaaabab \to abaabbb$

   • Рассмотрим неразрешимую критическую пару (baabbbbaaab, babbabbaaabb)

        Добавляем новое правило: $baaababaaab \to babbbaaabb$


 \textbf{Шаг 3}.

   • Рассмотрим неразрешимую критическую пару (abaabb, aabbaabb)

        Добавляем новое правило: $aabbaabb \to abaabb$

   • Рассмотрим неразрешимую критическую пару (bbaaabab, abaabbbb)

        Добавляем новое правило: $bbaaabab \to abaabbb$

    Обновляем правило: $babbaaabab \to babbb$ → $abaabbb \to babbb$

    Обновляем правило: $babbbaaabab \to babbb$ → $abaabbb \to babbb$

   • Рассмотрим неразрешимую критическую пару (babbaabb, bbabaabbb)

        Добавляем новое правило: $baabbb \to babbb$

    Удаляем правило: $abaabbb \to babbb$

    Обновляем правило: $babaabb \to abaabbb$ → $babaabb \to babbb$

    Удаляем правило: $bbaabbb \to abaabbb$

    Обновляем правило: $bbbaabb \to abaabbb$ → $bbbaabb \to babbb$

    Обновляем правило: $abaaabab \to abaabbb$ → $abaaabab \to babbb$

    Обновляем правило: $babbbaab \to abaabbb$ → $babbbaab \to babbb$

    Обновляем правило: $bbaaabab \to abaabbb$ → $bbaaabab \to babbb$

    Обновляем правило: $baabbbaaab \to babbaaabb$ → $babbbaaab \to babbaaabb$

   • Рассмотрим неразрешимую критическую пару (bbaabbaaab, aabaaabb)

        Добавляем новое правило: $aabaaabb \to bbaaabb$

    Обновляем правило: $baaababaaab \to babbbaaabb$ → $baaababaaab \to babbaaabbb$

   • Рассмотрим неразрешимую критическую пару (abaabbbaaab, babaaabb)

        Добавляем новое правило: $babbaaabb \to babaaabb$

    Обновляем правило: $babbbaaab \to babbaaabb$ → $babbbaaab \to babaaabb$

    Обновляем правило: $baaababaaab \to babbaaabbb$ → $baaababaaab \to babaaabbb$

   • Рассмотрим неразрешимую критическую пару (baabbbb, baabaabbb)

        Добавляем новое правило: $baaabab \to babbb$

    Удаляем правило: $abaaabab \to babbb$

    Удаляем правило: $bbaaabab \to babbb$

    Обновляем правило: $baaababaaab \to babaaabbb$ → $babaaabb \to babaaabbb$

   • Рассмотрим неразрешимую критическую пару (abaabbbaaab, bbabbaaabb)

        Добавляем новое правило: $bbabaaabb \to babaaabb$

   • Рассмотрим неразрешимую критическую пару (abaabbbaaab, bbbaaabb)

        Добавляем новое правило: $bbbaaabb \to babaaabb$

   • Рассмотрим неразрешимую критическую пару (baaababaabbb, babbbaaabbab)

        Добавляем новое правило: $ababaaabbab \to babbb$

   • Рассмотрим неразрешимую критическую пару (abaabbbaaab, ababbbaaabb)

        Добавляем новое правило: $ababaaabb \to babaaabb$

    Обновляем правило: $ababaaabbab \to babbb$ → $babaaabbab \to babbb$

   • Рассмотрим неразрешимую критическую пару (baaabbabaaab, baabbabbbaaabb)

        Добавляем новое правило: $baaabbabaaab \to babaaabb$


 \textbf{Шаг 4}.

   • Рассмотрим неразрешимую критическую пару (abaaabb, aabbaaabb)

        Добавляем новое правило: $aabbaaabb \to abaaabb$

   • Рассмотрим неразрешимую критическую пару (aabaaaabab, bbaaabbbb)

        Добавляем новое правило: $abbabb \to babbb$

    Обновляем правило: $aabbabb \to ababb$ → $babbb \to ababb$

   • Рассмотрим неразрешимую критическую пару (babaaaabab, babaaabbb)

        Добавляем новое правило: $babaaabb \to ababb$

    Обновляем правило: $ababbb \to babbb$ → $bbabb \to ababb$

    Обновляем правило: $abbabb \to babbb$ → $abbabb \to ababb$

    Обновляем правило: $baabab \to babbb$ → $baabab \to ababb$

    Обновляем правило: $baabbb \to babbb$ → $baabbb \to ababb$

    Обновляем правило: $babbab \to babbb$ → $babbab \to ababb$

    Обновляем правило: $bbabbb \to babbb$ → $bbabb \to ababb$

    Обновляем правило: $baaabab \to babbb$ → $baaabab \to ababb$

    Обновляем правило: $baabbab \to babbb$ → $baabbab \to ababb$

    Обновляем правило: $babaabb \to babbb$ → $babaabb \to ababb$

    Обновляем правило: $bbbaabb \to babbb$ → $bbbaabb \to ababb$

    Обновляем правило: $babbbaab \to babbb$ → $abaabb \to ababb$

    Обновляем правило: $bbbaaabb \to babaaabb$ → $bbbaaabb \to ababb$

    Обновляем правило: $ababaaabb \to babaaabb$ → $bbabb \to ababb$

    Обновляем правило: $babaaabbb \to babaaabb$ → $bbabb \to ababb$

    Обновляем правило: $babbaaabb \to babaaabb$ → $babbaaabb \to ababb$

    Обновляем правило: $babbbaaab \to babaaabb$ → $ababbaaab \to ababb$

    Обновляем правило: $bbabaaabb \to babaaabb$ → $bbabb \to ababb$

    Обновляем правило: $babaaabbab \to babbb$ → $bbabb \to ababb$

    Обновляем правило: $baaabbabaaab \to babaaabb$ → $baaabbabaaab \to ababb$

   • Рассмотрим неразрешимую критическую пару (baabbabbb, baaabbab)

        Добавляем новое правило: $baaabbab \to ababb$

    Обновляем правило: $aababb \to bbabb$ → $aababb \to ababb$

    Удаляем правило: $abbabb \to ababb$

    Обновляем правило: $aabaabb \to bbaabb$ → $ababb \to bbaabb$

    Удаляем правило: $babaabb \to ababb$

    Обновляем правило: $aabbaabb \to abaabb$ → $aabbaabb \to ababb$

    Удаляем правило: $baaabbabaaab \to ababb$

   • Рассмотрим неразрешимую критическую пару (abaabbaaab, aabbaaabb)

        Добавляем новое правило: $abaaabb \to ababb$

    Удаляем правило: $bbbaabb \to ababb$

    Обновляем правило: $aabaaabb \to bbaaabb$ → $ababb \to bbaaabb$

    Удаляем правило: $aabbaabb \to ababb$

    Удаляем правило: $babaaabb \to ababb$

    Обновляем правило: $aabbaaabb \to abaaabb$ → $aabbaaabb \to ababb$


 \textbf{Шаг 5}.

   • Рассмотрим неразрешимую критическую пару (baaababb, baaabbb)

        Добавляем новое правило: $baaabbb \to ababb$

    Удаляем правило: $bbbaaabb \to ababb$

    Удаляем правило: $aabbaaabb \to ababb$

    Удаляем правило: $babbaaabb \to ababb$


 \textbf{Шаг 6}.

Неразрешимых критических пар не осталось. 

Эквивалентная система:

$aaaa \to a$

$bbbb \to abab$

$babab \to ababb$

$babbb \to ababb$

$bbabb \to ababb$

$bbbab \to babb$

$aababb \to ababb$

$abaabb \to ababb$

$baabab \to ababb$

$baabbb \to ababb$

$babbab \to ababb$

$bbaabb \to ababb$

$abaaabb \to ababb$

$baaabab \to ababb$

$baaabbb \to ababb$

$baabbab \to ababb$

$babbaab \to baabb$

$bbaaabb \to ababb$

$baaabbab \to ababb$

$ababbaaab \to ababb$

$baabbaaab \to baaabb$

\section{Тестирование}

В качестве инвариантов для метаморфного тестирования были рассмотрены:
\begin{enumerate}[topsep=0pt,parsep=-5pt]
    \item Последняя буква в исходном слове и в слове, полученном при переписывании в рамках SRS, не меняется.
    \item Число подряд идущих букв $a$ в конце слова по модулю 3 совпадает.
    \item $(|w|_{ba} + |w|_{bb} + |w|_b)$ по модулю 2 не возрастает.
    \item $|w|_{ab} + |w|_{bb} - |w|_b$ не возрастает.
\end{enumerate}

\end{document}
